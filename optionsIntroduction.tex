An \textbf{option} gives the right to buy or sell something in the future at a fixed price. The value of an option comes from price uncertainty: if the future price is high the option to buy at a fixed lower price is valuable. An option to sell at a fixed price hedges the risk that the price will go down.

Option prices embed predictions about other prices. From a set of option prices it is possible to know the entire probability distribution of future prices for the underlying asset. We can know the (market) probability that the S\&P will crash to 900 next year, or the probability that crude oil will be above \$100 in 5 years time. If we disagree with these market probabilities we can construct option portfolios that bet against them.

We can also bet on characteristics of the distribution. For example we might believe that the market underprices extreme events and so the tails of the market distribution are too small. Or that there is more downside risk than upside risk.  Or anything else about the probability distribution of the future price. We can construct option portfolios to bet on these views. 

Options allows a more sophisticated type of investment. Usually investors will start with some view about the future price distribution of an asset. If the average of their expected distribution is greater than the forward price they think it's a good idea to buy. If it's less they sell. Maybe they make some adjustment for risk. This is almost always wrong. The optimal portfolio should start by comparing the investor's beliefs with the market predicted distribution. Almost always the optimal portfolio involves options.

Investment without options is basically indefensible, but quite understandable. Option pricing uses complicated mathematics and is not accessible to most investors. Even professional investment managers will have a murky understanding of options pricing and basic option structures, but not enough to sensibly integrate them into a portfolio. 

This course attempts to give a reasonably sophisticated understanding of pricing and using options without the really tricky math. We develop the intuition of the pricing, but keep the focus on practicalities. The goal is to be able to construct and manage an option portfolio to express a view about the distribution of future prices. 

