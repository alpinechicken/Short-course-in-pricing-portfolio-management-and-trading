\documentclass{beamer}
\usepackage{beamerthemesplit} % new
\begin{document}
\title{How the market predicts the future}
\author{Joseph Clark}
\date{April 23 2010}
 
\frame{\titlepage}
 
%\frame{\frametitle{Table of contents}\tableofcontents}
 
\frame{\frametitle{The old and the new}

Basic idea: Instead of being and selling \textbf{prices} we now buy and sell \textbf{probabilities}.

\begin{block}{Traditional approach to investing}
\begin{itemize}
\item Market price of a stock is \$100
\item You think the fair price is \$110  dollars
\item You buy the stock
\end{itemize}
\end{block}


}

\frame{

\begin{block}{Modern approach to investing}
\begin{exampleblock}{Probability trading}
\begin{itemize}
\item Market \textbf{probability} of the price being above \$105 in one year's time is 40\% (determined from option prices)
\item \textbf{Buy the probability}.
\end{itemize}
\end{exampleblock}

\begin{exampleblock}{Volatility trading}
\begin{itemize}
\item Market thinks the volatility (standard deviation) of price returns on the S\&P 500  will be 17\% over the next month.
\item Buy or sell a \textbf{volatility swap}
\item Trading volatility is a compound trade on probabilities
\end{itemize}
\end{exampleblock}
\end{block}
}


\frame{

\begin{block}{Advantage of modern approach}
\begin{itemize}
\item \textbf{More flexible}: construct targeted exposures across the full range of possible outcomes
\item \textbf{At least as good as traditional approach}: forward price is the market threshold for 50\% probability
\end{itemize}
\end{block}

}
 
%\section{Section no.1}
\frame{\frametitle{Fundamentals: The price-space-time continuum}
 
 What are these probabilities defined over?
 
\begin{itemize}
\item A price is specific to a \textbf{location} and a \textbf{time}. The map of prices throughout time and space is the \textbf{price-space-time} continuum.\\
\item The surface of price-space-time includes \textbf{spot} and \textbf{forward} prices and is connected by \textbf{time and space arbitrage}
    \begin{itemize}
    \item Cash and carry arbitrage: buy and hold to sell later\
    \item Transport arbitrage: buy at one place and sell at another place\\
    \end{itemize}
\end{itemize}
 
\begin{exampleblock}{Price space time for oil}
 
The price of a barrel of oil is different delivered today or next week; Queensland or Texas. But the difference between prices is bounded by transport price and the cost of borrowing money and storing oil.
 
[figure of price at L1 and L2]
\end{exampleblock}
 
%\end{itemize}
}
 
 
\frame{\frametitle{Subjective probabilities over price}
 
Each slice of price-space-time has a \textbf{subjective probability distribution}
 
\begin{itemize}
\item Made up by whoever is predicting
\item Defines the probability of the price being at any point between 0 and $\infty$ in the future
\item Probability of range is area under the curve.\\
\item Expected value under the probability distribution is the future fair value.\\
\end{itemize}
 
[Pic of of Prob dist at three times]
 
 
\frame{\frametitle{Market probabilities over price}
 
Each point in price-space-time has a \textbf{market probability distribution} (risk neutral probability distribution)
 
\begin{itemize}
\item Observable and tradable if there is a vanilla options market
\item Defines the probability of the price being at any point between 0 and $\infty$ in the future
\item Probability of range is area under the curve.\\
\item Expected value under the probability distribution is the \textbf{forward price}.\\
\end{itemize}
 
[Pic of of Prob dist at three times]
 
\begin{exampleblock}{Price of oil next year in Texas}
 
[figure of price at L1 and L2]
 
\end{exampleblock}
}
 
\frame{\frametitle{Trading the probability distribution}
 
\begin{block}{How to trade probabilities}
\begin{itemize}
\item Pick a slice of the price-space-time continuum (e.g. soy beans in May 2013 in the US)
\item Construct your subjective probabilities
\item Calculate the market probabilities
\item Trade your distribution against the market
\end{itemize}
\end{block}
 
[pic of subjective and risk neutral]
 
}
 
\frame{\frametitle{Trading the forward distribution (1): spot vs forward}
 
\begin{block}{Punting long/short against the forward price}
\begin{itemize}
\item Determine the average value under your subjective distribution\\
\item Compare to the forward price
\begin{itemize}
\item If greater than the forward price: buy the forward
\item If less than the forward price: sell the forward
\end{itemize}
\end{itemize}
\end{block}

 
}
 
\frame{\frametitle{Trading the forward distribution (2): Binaries}
 
A binary is a security that pays \$1 if the price is above some value $K$ and \$0 otherwise
 
[pic of binary]
 
The market price of a binary is also the market probability.
 
\begin{block}{Buying or selling probabilities}
\begin{itemize}
\item Determine the probability of the price being greater than some value, e.g. P(S\&P 500 $>$ 1500 in one year)\\
\begin{itemize}
\item If greater market probability: buy a binary
\item If less than market probability: sell a binary
\end{itemize}
\end{itemize}
\end{block}
 
}
 
\frame{\frametitle{Trading the forward distribution (3): Call and put options}
 
\begin{itemize}
\item A \textbf{call option} pays $\max (S(T)-K,0)$
\item A \textbf{put option} pays $\max (K-S(T),0)$
\end{itemize}
 
[pic of call and puts]
 
 
\begin{block}{Punting options}
\begin{itemize}
\item Determine the expected value of the option under your probability distribution (multiple payoff by probability)\\
\begin{itemize}
\item If greater than market option price: buy option
\item If less than market option price: sell option
\end{itemize}
\end{itemize}
\end{block}
 
[Pic of option payoff superimposed on probability]
 
}
 
 
 
\frame{\frametitle{Trading the forward distribution (4): Trading the price path}
 
The \textbf{variance} of returns is expected value over some time is \textbf{the expected squared deviation from the average}:
 
\[V \equiv E[(\mbox{price return(t)} - \mbox{average price return})^2] \]
 
 
\begin{itemize}
\item Determine the expected variance under your probability distribution of the price at each point in time\\
\item Determine the market price of variance under the market's probability distributions
\begin{itemize}
\item If expected variance $>$ market variance. Buy variance swap
\item If expected variance $<$ market variance. Sell variance swap
\end{itemize}
\end{itemize}
 
 }
 
 \frame{
 
 \begin{exampleblock}{A volatility swap}
 The 30 day volatility swap rate for the S\&P 500 index is $K_{var} = 17.2$. We \textbf{buy} \$10,000,000 notional the swap. After 30 days the realised  volatility of log prices is 22.5. The payoff of the swap is:
 
 \[\mbox{volatility swap payoff} =  \mbox{(realized volatility - swap)* notional } = (22.5-17.5)*10m = 40m\]
 
 \end{exampleblock}
 
  E.g of variance swap. Pic of VIX index
 
 }
 
 \frame{
 

 \begin{block}{Other ways to trade variance}
\begin{itemize}
\item Volatility swap\\
\item Trade delta hedged options\\
\end{itemize}
 \end{block}
 
 
  \begin{block}{Other statistical characteristics to trade}
\begin{itemize}
\item Skewness (asymmetry between up and down moves)\\
\item Kurtosis (fat tails)\\
\end{itemize}
 \end{block}
 
 }
 
 
\frame{\frametitle{Practical probability trading examples}

\begin{exampleblock}{You don't know where the price of AUD/USD will go but you're positive that it won't go above 1.25 in three months.  }


\begin{itemize}
\item Sell three month binary at 1.25
\item Sell three month call options at 1.25
\end{itemize}
\end{exampleblock}
 }
 
 \frame{
 
 \begin{exampleblock}{You don't know where the price of oil will go but you're sure that it will be at least +/- 10\% by next week }


\begin{itemize}
\item Buy puts at 90\% of the current price
\item Buy calls at 110\% of the current price
\item Buy a volatility or variance swap (depending on swap rate)
\end{itemize}
\end{exampleblock}
 
 }
 
 \frame{
 \begin{exampleblock}{You think the market is much too worried about a crash in equities}


\begin{itemize}
\item (Traditional approach:) Buy! Buy! Buy! Then borrow and buy some more.
\item (Modern approach) Sell put options on the S\&P significantly below the current price
\item Buy binaries below the current price
\item Sell variance/volatiltiy
\end{itemize}
\end{exampleblock}

}

\frame{
 
\begin{exampleblock}{You think the market is not worried enough about a crash in equites}


\begin{itemize}
\item (Traditional approach): Sell all your equities. Stock up on canned beans and ammunition.
\item (Modern approach): Buy put options on the S\&P significantly below the current price
\item Sell binaries below the current price
\item Buy variance/volatility
\end{itemize}
\end{exampleblock}
}
 
 
\frame{
 \begin{exampleblock}{The current RBA cash rate is 5\%. You think that the RBA will definitely raise rates 25 basis points but definitely not more}


\begin{itemize}
\item (Traditional approach): Buy cash rate futures.
\item (Modern approach): Buy call options on the rate at 5\%
\item Sell call options on the rate at 5.25\%
\item Same with binaries
\end{itemize}
\end{exampleblock}

[picture]

}
 
\frame{
\begin{exampleblock}{You need to build a risk model to determine the 1\% VaR for a long only portfolio}


\begin{itemize}
\item (Traditional approach) Calculate the historical 1\% tail for the assets in the portfolio.
\item (Modern approach) For each asset in the portfolio find the market price range with a 1\% probability. Calculate the value of the portfolio at these prices.
\begin{itemize}
\item This will be a worst case estimate since all the assets are unlikely to all move against you.
\item The combined value of put options at these prices is the price you need to pay to ensure the portfolio never falls below this worst case number.
\end{itemize}
\end{itemize}
\end{exampleblock}

}

\frame{
 
\begin{exampleblock}{How much riskier is oil than gold?}


\begin{itemize}
\item Depends on what you mean by risky.
\item (Traditional approach): Guess/ look at historical prices / build factor model
\item (Modern approach): Ask the market
\begin{itemize}
\item Calculate market value of volatility swap on oil and gold. Take ratio.
\item Calculate the market probability of a 50\% drop in the assets over some time horizon. Take ratio
\item Etc.
\end{itemize}
\end{itemize}
\end{exampleblock} 

}
 
 
\frame{
\begin{exampleblock}{You think that equites are good value because the equity risk premium is high.}


\begin{itemize}
\item (Traditional approach) Buy some equites. Go fishing.
\item (Modern approach) If you think the risk premium represents market fears of volatility sell a volatility swap (or equivalent). Go fishing.
\end{itemize}
\end{exampleblock}
 
}
 
\end{document}

Other stuff:

Fixed income. (rolldown)
QIC stories


\begin{center}
\begin{figure}
\includegraphics[scale=0.3]{ab12}
\caption{Pick two cards}
\end{figure}
\end{center}


 